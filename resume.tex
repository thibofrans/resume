%%%%%%%%%%%%%%%%%
% This is an sample CV template created using altacv.cls
% (v1.6.4, 13 Nov 2021) written by LianTze Lim (liantze@gmail.com). Now compiles with pdfLaTeX, XeLaTeX and LuaLaTeX.
%
%% It may be distributed and/or modified under the
%% conditions of the LaTeX Project Public License, either version 1.3
%% of this license or (at your option) any later version.
%% The latest version of this license is in
%%    http://www.latex-project.org/lppl.txt
%% and version 1.3 or later is part of all distributions of LaTeX
%% version 2003/12/01 or later.
%%%%%%%%%%%%%%%%

%% Use the "normalphoto" option if you want a normal photo instead of cropped to a circle
% \documentclass[10pt,a4paper,normalphoto]{altacv}



\documentclass[10pt,a4paper,ragged2e,withhyper]{altacv}
%% AltaCV uses the fontawesome5 and packages.
%% See http://texdoc.net/pkg/fontawesome5 for full list of symbols.

% Change the page layout if you need to
\geometry{left=2.5cm,right=2.5cm,top=3cm,bottom=3cm,columnsep=1.2cm}

% The paracol package lets you typeset columns of text in parallel
\usepackage{paracol}

% Change the font if you want to, depending on whether
% you're using pdflatex or xelatex/lualatex
\ifxetexorluatex
  % If using xelatex or lualatex:
  \setmainfont{Roboto Slab}
  \setsansfont{Lato}
  \renewcommand{\familydefault}{\sfdefault}
\else
  % If using pdflatex:
  \usepackage[rm]{roboto}
  \usepackage[defaultsans]{lato}
  \usepackage{sourcesanspro}
  \renewcommand{\familydefault}{\sfdefault}
\fi

% Change the colours if you want to
\definecolor{SlateGrey}{HTML}{2E2E2E}
\definecolor{LightGrey}{HTML}{666666}
\definecolor{DarkPastelRed}{HTML}{450808}
\definecolor{PastelRed}{HTML}{8F0D0D}
\definecolor{GoldenEarth}{HTML}{E7D192}
\definecolor{Light}{HTML}{93A9B4}
\definecolor{Medium}{HTML}{607D8B}
\definecolor{Dark}{HTML}{40535D}
\colorlet{name}{Dark}
\colorlet{tagline}{Dark}
\colorlet{heading}{Dark}
\colorlet{headingrule}{Light}
\colorlet{subheading}{Medium}
\colorlet{accent}{Light}
\colorlet{emphasis}{Dark}
\colorlet{body}{LightGrey}

% Change some fonts, if necessary
\renewcommand{\namefont}{\Huge\sffamily\bfseries}
\renewcommand{\personalinfofont}{\footnotesize}
\renewcommand{\cvsectionfont}{\LARGE\sffamily\bfseries}
\renewcommand{\cvsubsectionfont}{\large\bfseries}


% Change the bullets for itemize and rating marker
% for \cvskill if you want to
\renewcommand{\itemmarker}{{\small\textbullet}}
\renewcommand{\ratingmarker}{\faCircle}

%% Use (and optionally edit if necessary) this .cfg if you
%% want to use an author-year reference style like APA(6)
%% for your publication list
\input{pubs-authoryear.cfg}

%% Use (and optionally edit if necessary) this .cfg if you
%% want an originally numerical reference style like IEEE
%% for your publication list
% \input{pubs-num.cfg}

%% sample.bib contains your publications
%\addbibresource{sample.bib}

%TODO: use class variables
\renewcommand{\makecvheader}{%
	\begingroup
	\hfill%
	\begin{minipage}{\dimexpr\linewidth}%
		\centering%
		{\namefont\color{name}Thibo Frans\par}
		\medskip
    {\textrm{\taglinefont\color{tagline}Cirriculum Vitae}\par}
		\bigskip
    {\personalinfofont {
        % Not all of these are required!
        \email{thibo.frans@live.be}
        \phone{0468 94 08 68}
        \mailaddress{Tarwestraat 14/001 9000 Gent}
        \location{Gent, BELGIUM}
        \homepage{thibofrans.be}
        %\twitter{@twitterhandle}
        %\linkedin{your_id}
        \github{thibofrans}
        %\orcid{0000-0000-0000-0000}
        %% You can add your own arbitrary detail with
        %% \printinfo{symbol}{detail}[optional hyperlink prefix]
        % \printinfo{\faPaw}{Hey ho!}[https://example.com/]
        %% Or you can declare your own field with
        %% \NewInfoFiled{fieldname}{symbol}[optional hyperlink prefix] and use it:
        % \NewInfoField{gitlab}{\faGitlab}[https://gitlab.com/]
        % \gitlab{your_id}
        %%
        %% For services and platforms like Mastodon where there isn't a
        %% straightforward relation between the user ID/nickname and the hyperlink,
        %% you can use \printinfo directly e.g.
        % \printinfo{\faMastodon}{@username@instace}[https://instance.url/@username]
        %% But if you absolutely want to create new dedicated info fields for
        %% such platforms, then use \NewInfoField* with a star:
        % \NewInfoField*{mastodon}{\faMastodon}
        %% then you can use \mastodon, with TWO arguments where the 2nd argument is
        %% the full hyperlink.
        % \mastodon{@username@instance}{https://instance.url/@username}
      }\par}
	\end{minipage}\hfill%
	\par%
	\endgroup\medskip
}

\renewcommand{\cvsection}[2][]{%
\nointerlineskip\bigskip%  %% bugfix in v1.6.2
\ifstrequal{#1}{}{}{\marginpar{\vspace*{\dimexpr1pt-\baselineskip}\raggedright\input{#1}}}%
{\color{heading}\cvsectionfont#2}\\[-1ex]%
{\color{headingrule}\rule{\linewidth}{2pt}\par}\medskip
}


\renewcommand{\taglinefont}{\large\itshape}

\begin{document}
%\name{Thibo Frans}
%\tagline{Cirriculum Vitae}
%% You can add multiple photos on the left or right
%\photoR{2.8cm}{Globe_High}
% \photoL{2.5cm}{Yacht_High,Suitcase_High}

%\personalinfo{%
% Not all of these are required!
%\email{thibo.frans@live.be}
%\phone{0468 94 08 68}
%\mailaddress{Tarwestraat 14/001 9000 Gent}
%\location{Gent, BELGIUM}
%\homepage{thibofrans.be}
%\twitter{@twitterhandle}
%\linkedin{your_id}
%\github{thibofrans}
%\orcid{0000-0000-0000-0000}
%% You can add your own arbitrary detail with
%% \printinfo{symbol}{detail}[optional hyperlink prefix]
% \printinfo{\faPaw}{Hey ho!}[https://example.com/]
%% Or you can declare your own field with
%% \NewInfoFiled{fieldname}{symbol}[optional hyperlink prefix] and use it:
% \NewInfoField{gitlab}{\faGitlab}[https://gitlab.com/]
% \gitlab{your_id}
%%
%% For services and platforms like Mastodon where there isn't a
%% straightforward relation between the user ID/nickname and the hyperlink,
%% you can use \printinfo directly e.g.
% \printinfo{\faMastodon}{@username@instace}[https://instance.url/@username]
%% But if you absolutely want to create new dedicated info fields for
%% such platforms, then use \NewInfoField* with a star:
% \NewInfoField*{mastodon}{\faMastodon}
%% then you can use \mastodon, with TWO arguments where the 2nd argument is
%% the full hyperlink.
% \mastodon{@username@instance}{https://instance.url/@username}
%}

\makecvheader
%% Depending on your tastes, you may want to make fonts of itemize environments slightly smaller
% \AtBeginEnvironment{itemize}{\small}

%% Set the left/right column width ratio to 6:4.
\columnratio{0.6}

% Start a 2-column paracol. Both the left and right columns will automatically
% break across pages if things get too long.
\begin{paracol}{2}

	\cvsection{Opleiding}

	\cvevent{Graduaat Programmeren}{Hogeschool Gent}{Sept 2021 -- heden}{}

	\divider

	\cvevent{Bachelor Informatica}{Universiteit Gent}{Sept 2020 -- Aug 2021}{}

	\divider

	\cvevent{Bachelor Fysica \& Sterrenkunde}{Universiteit Gent}{Sept 2016 -- Aug 2018}{}

	\divider

	\cvevent{ASO Wetenschappen--Wiskunde}{Examencomissie Secundair Onderwijs}{Sept 2016 -- Okt 2018}{}

	% \divider
	\cvsection{Ervaring}

	\cvevent{Winkel Medewerker}{Carrefour Market Groenevallei/Mariakerke}{Okt 2018 -- Aug 2020}{Gent}
	\begin{itemize}
		\item Verantwoordelijk voor the dagelijks vers groenten en fruit.
	\end{itemize}

	\cvsection{Projecten}

	\cvevent{Polis}{Universiteit Gent}{2021}{Gent}
	Schoolopdracht: een simulatie van een stad (denk aan SimCity) geschreven in Java en JavaFX.

	%% Switch to the right column. This will now automatically move to the second
	%% page if the content is too long.
	\switchcolumn

	\cvsection{Vaardigheden}

	\cvsubsection{Programmeertalen}

	\cvtag{Java}
	\cvtag{C\#}
	\cvtag{Python}
	\cvtag{JS}
	\cvtag{Bash}

	\divider

	\cvtag{HTML}
	\cvtag{CSS}
	\cvtag{Markdown}
	\cvtag{\LaTeX} \\
	\cvtag{AsciiDoc}

	\medskip

	\cvsubsection{Technologie\"en}

	\cvtag{JavaFX}
	\cvtag{JUnit}
	\cvtag{JDBC}
	\cvtag{Git}\\
	\cvtag{MySQL}
	\cvtag{PostgreSQL}
	\cvtag{Linux}

	\medskip

	\cvsubsection{Sterktes}

	\cvtag{Probleemoplossend denken} \\
	\cvtag{Stipt}
	\cvtag{Flexibel}

	\cvsection{Talen}

	\cvsubsection{Natuurlijke talen}

	\cvskill{Nederlands}{5}
	\cvskill{English}{4.5}
	\cvskill{Fran\c{c}ais}{2}

	\medskip

	\cvsection{Certificaten}

	\cvachievement{\faIcon{database}}{Oracle}{Database Foundations}

	\medskip

	%\cvachievement{\faIcon{server}}{CompTIA}{A\textsuperscript{+}}



	% use ONLY \newpage if you want to force a page break for
	% ONLY the current column
	%\newpage
\end{paracol}


\end{document}
